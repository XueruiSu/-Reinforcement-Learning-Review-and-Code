%!TEX root = 。。/。。/csuthesis_main。tex
\chapter{最优化理论的发展历程}
\cite{陈宝林2005最优化理论与算法}最优化理论与算法是一个重要的数学分支,它所研究的问题是讨论在众多的方案中什么样的方案最优以及怎样找出最优方案。这类问题普遍存在。例如,工程设计中怎样选择设计参数,使得设计方案既满足设计要求又能降低成本,资源分配中, 怎样分配有限资源,使得分配方案既能满足各方面的基本要求,又能获得好的经济效益; 生产计划安排中, 选择怎样的计划方案才能提高产值和利润; 原料配比问题中,怎样确定各种成分的比例, 才能提高质量,降低成本; 城建规划中, 怎样安排工厂、机关、学校、商店、医院、住户和其他 单位的合理布局, 才能方便群众,有利于城市各行各业的发展; 农田规划中,怎样安排各种 农作物的合理布局, 才能保持高产稳产, 发挥地区优势;军事指挥中,怎样确定最佳作战方案,才能有效地消灭敌人, 保存自己; 在人类活动的各个领域中, 诸如 此类,不胜枚举。最优化这一数学分支,正是为这些问题的解决提供理论基础和求解方 法,它是一门应用广泛、实用性强的学科。

最优化是个古老的课题。 长期以来,人们对最优化问题进行着探讨和研究。 早在 17世纪, 英国科学家 Newton 发明微积分的时代,就已提出极值问题,后来又出现 Lagrange乘数法。 1847 年法国数学家 Cauchy 研究了函数值沿什么方向下降最快的问题,提出最速下 降法。 1939 年前苏联数学家坎托罗维奇提出了解决下料问题和运输问题这两种线 性规划问题的求解方法。 人们关于最优化问題的研究工作,随着历史的发展不断深入。 但 是,任何科学的进步,都受到历史条件的限制,直至 20 地纪 30 年代,最优化这个古老课题 并末形成独立的有系统的学科。

20 世纪 40 年代以来, 由于生产和科学研究突飞猛进地发展, 特别是电子计算机广泛的应用,使最优化问题的研究不仅成为一种迫切需要,而且有了求解的有力工具。因此 最优化理论和算法飞速发展起来,形成一个新的学科。至今已出现线性规划、整数规划、非 线性规划、几何规划、动态规划、随机规划、网络流等许多分支。 最优化理论和算法在实际 应用中正在发挥越来越大的作用。

最优化理论自诞生以来,在各个领域的应用非常广泛,尤其在人工智能领域,在控制、预测、搜索、估计等任务中发挥了至关重要的作用。本文基于本学期学到的知识,站在优化的角度上,梳理了人工智能领域控制、预测、搜索三大任务情景下最优化算法的设计与使用。可以说,最优化的问题形式在这三类任务情景中均发挥了或是作为算法重要部分或是引导出更简单近似方法的作用。
