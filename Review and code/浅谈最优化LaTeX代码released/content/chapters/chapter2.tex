%!TEX root = ../../csuthesis_main.tex
\chapter{最优化基本概念}
\vspace{-0.8cm}
\section{最优化问题}
\vspace{-0.3cm}
\subsection{最优化问题定义}
\cite{boyd2004convex}数学优化问题或者说优化问题可以写成如下形式:
\vspace{-0.4cm}
$$\begin{array}{ll}
\operatorname{\textbf{minimize}} & f_{0}(x) \\ 
\textbf{subject to}& f_{i}(x) \leqslant b_{i}, \quad i=1, \cdots, m
\end{array}$$

\vspace{-0.4cm}
这里, 向量 $x=\left(x_{1}, \cdots, x_{n}\right)$ 称为问题的\textbf{优化变量}, 函数 $f_{0}: \mathbf{R}^{n} \rightarrow \mathbf{R}$ 称为\textbf{目标函数}, 函数 $f_{i}: \mathbf{R}^{n} \rightarrow \mathbf{R},\  i=1, \cdots, m$, 被称为 (不等式) \textbf{约束函数}, 常数 $b_{1}, \cdots, b_{m}$ 称为\textbf{约束上限}或者\textbf{约束边界}。如果在所有满足约束的向量中向量 $x^{\star}$ 对应的目标函数值最小: 即对于任意满足约束 $f_{1}(z) \leqslant b_{1}, \cdots, f_{m}(z) \leqslant b_{m}$ 的向量 $z$,\  有 $f_{0}(z) \geqslant f_{0}\left(x^{\star}\right)$,\  那么称 $x^{\star}$ 为上述优化问题的\textbf{最优解}或者\textbf{解},$f_{0}\left(x^{\star}\right)$为\textbf{最优值}。
\vspace{-0.5cm}
\subsection{最优化问题分类}
使用不同的分类指标可以将最优化问题分为不同的类别,如表\ref{fenlei}所示:
% Table generated by Excel2LaTeX from sheet 'Sheet1'
\begin{table}[htbp]
  \centering
  \caption{最优化问题分类方法}
    \begin{tabular}{ccc}
    \toprule
    \toprule
    划分标准  & \multicolumn{2}{c}{问题分类} \\
    \midrule
    \multirow{3}[2]{*}{有无约束} & \multicolumn{2}{c}{无约束问题} \\
          & \multicolumn{2}{c}{等式约束问题} \\
          & \multicolumn{2}{c}{不等式约束问题} \\
    \midrule
    \multirow{3}[2]{*}{目标函数和约束函数的线性程度} & \multicolumn{2}{c}{线性规划问题} \\
          & \multicolumn{2}{c}{二次规划问题} \\
          & \multicolumn{2}{c}{非线性规划问题} \\
    \midrule
    \multirow{2}[2]{*}{目标函数和可行域的凸性} & \multicolumn{2}{c}{凸优化问题} \\
          & \multicolumn{2}{c}{非凸优化问题} \\
    \midrule
    \multicolumn{1}{c}{\multirow{2}[2]{*}{目标函数和约束函数的解析性质}} & \multicolumn{2}{c}{光滑优化问题} \\
          & \multicolumn{2}{c}{非光滑优化问题} \\
    \midrule
    \multirow{5}[4]{*}{可行域中可行点的个数} & \multicolumn{2}{c}{连续优化问题} \\
\cmidrule{2-3}          & \multirow{4}[2]{*}{离散优化问题} & 0-1规划问题 \\
          &       & 整数规划问题 \\
          &       & 0-1混合规划问题 \\
          &       & 混合整数规划问题 \\
    \midrule
    \multirow{2}[2]{*}{模型参数的确定性} & \multicolumn{2}{c}{确定型规划问题} \\
          & \multicolumn{2}{c}{随机规划问题} \\
    \bottomrule
    \bottomrule
	\end{tabular}%
  \label{fenlei}%
\end{table}%

虽然以上优化问题均在相应领域有很广泛地应用,但作为初学者,我们主要学习目标函数和约束函数均连续可微的确定型规划问题,即非线性最优化问题。

\section{凸集}
在最优化问题的理论分析中,常用到凸集和凸函数的概念,所以下面我们简单地给出相关的定义。

集合 $C$ 被称为凸集, 如果 $C$ 中任意两点间的线段仍然在 $C$ 中, 即对于任意 $x_{1}, x_{2} \in C$ 和满足 $0 \leqslant \theta \leqslant 1$ 的 $\theta$ 都有
$$
\theta x_{1}+(1-\theta) x_{2} \in C .
$$

粗略地, 如果集合中的每一点都可以被其他点沿着它们之间一条无阻碍的路径看见, 那 么这个集合就是凸集。所谓无阻碍, 是指整条路径都在集合中。由于仿射集包含穿过集 合中任意不同两点的整条直线, 任意不同两点间的线段自然也在集合中。因而仿射集是 凸集。图2.2显示了 $\mathbf{R}^{2}$ 空间中一些简单的凸和非凸集合。

我们称点 $\theta_{1} x_{1}+\cdots+\theta_{k} x_{k}$ 为点 $x_{1}, \cdots, x_{k}$ 的一个凸组合, 其中 $\theta_{1}+\cdots+\theta_{k}=1$ 并且 $\theta_{i} \geqslant 0, i=1, \cdots, k$ 。与仿射集合类似, 一个集合是凸集等价于集合包含其中所有点的凸组合。点的凸组合可以看做它们的混合或加权平均, $\theta_{i}$ 代表混合时 $x_{i}$ 所占的份数。

我们称集合 $C$ 中所有点的凸组合的集合为其凸包, 记为 $\operatorname{conv} C$ :
$$
\operatorname{conv} C=\left\{\theta_{1} x_{1}+\cdots+\theta_{k} x_{k} \mid x_{i} \in C, \theta_{i} \geqslant 0, i=1, \cdots, k, \theta_{1}+\cdots+\theta_{k}=1\right\} .
$$

顾名思义, 凸包 $\operatorname{conv} C$ 总是凸的。它是包含 $C$ 的最小的凸集。也就是说, 如果 $B$ 是 包含 $C$ 的凸集, 那么 $\operatorname{conv} C \subseteq B$ 。

与凸集相关的概念还有仿射集合,凸锥,半正定锥($S^{n}_{+}$,在微分几何中,三维欧氏空间下,$S^{2}$对应二维流形),正定锥($S^{n}_{++}$),超平面,Euclid球,范数球,范数锥,多面体,单纯形等集合的定义。随着学习的深入将再进一步的补充。

\section{凸函数}
函数 $f: \mathbf{R}^{n} \rightarrow \mathbf{R}$ 是凸的, 如果 $\operatorname{\textbf{dom}} f$ 是凸集, 且对于任意 $x, y \in \operatorname{\textbf{dom}} f$ 和任 意 $0 \leqslant \theta \leqslant 1$, 有
$$
f(\theta x+(1-\theta) y) \leqslant \theta f(x)+(1-\theta) f(y)
$$

从几何意义上看, 上述不等式意味着点 $(x, f(x))$ 和 $(y, f(y))$ 之间的线段, 即 从 $x$ 到 $y$ 的弦, 在函数 $f$ 的图像上方 。称函数 $f$ 是严格凸的, 如果 上式中的不等式当 $x \neq y$ 以及 $0<\theta<1$时严格成立。称函数 $f$ 是凹的, 如果函数 $-f$ 是凸的;称函数 $f$ 是严格凹的, 如果$-f$严格凸。

对于仿射函数,上述不等式总成立。因此所有仿射函数 (包括线性函数)是既凸且凹的。反之,若某个函数是既凸又凹的, 则其是仿射函数。

函数是凸的, 当且仅当其在与其定义域相交的任何直线上都是凸的。换言之, 函数 $f$ 是凸的, 当且仅当对于任意 $x \in \operatorname{\textbf{dom}} f$ 和任意向量 $v$, 函数 $g(t)=f(x+t v)$ 是凸 的 (其定义域为 $\{t \mid x+t v \in \operatorname{\textbf{dom}} f\}$ )。这个性质非常有用, 因为它容许我们通过将函 数限制在直线上来判断其是否是凸函数。


