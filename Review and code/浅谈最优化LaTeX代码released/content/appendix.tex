%!TEX root = ../csuthesis_main.tex
% \begin{appendixs} % 无章节编号
\chapter{附录代码}
%\begin{algorithm}[h]
%    \caption{堆溢出检测算法}\label{alg:ovf}
%    \begin{algorithmic}[1]
%        \IF {$\beta \in \mathbb{N^{*}} \land \Delta_\beta = \Delta_{\beta - 1} \land \beta < S$}
%            \STATE 正常写入
%        \ELSIF {$\beta \in \mathbb{N^{*}} \land \Delta_\beta \neq \Delta_{\beta - 1} \land \beta \geq S$}
%            \STATE 发生堆溢出
%        \ENDIF
%    \end{algorithmic}
%\end{algorithm}
\section{SARSA算法$pytorch$代码}
\begin{lstlisting}
import matplotlib.pyplot as plt
import numpy as np
from tqdm import tqdm  # tqdm是显示循环进度条的库


class CliffWalkingEnv:
    def __init__(self, ncol, nrow):
        self.nrow = nrow
        self.ncol = ncol
        self.x = 0  # 记录当前智能体位置的横坐标
        self.y = self.nrow - 1  # 记录当前智能体位置的纵坐标

    def step(self, action):  # 外部调用这个函数来改变当前位置
        # 4种动作, change[0]:上, change[1]:下, change[2]:左, change[3]:右。坐标系原点(0,0)
        # 定义在左上角
        change = [[0, -1], [0, 1], [-1, 0], [1, 0]]
        self.x = min(self.ncol - 1, max(0, self.x + change[action][0]))
        self.y = min(self.nrow - 1, max(0, self.y + change[action][1]))
        next_state = self.y * self.ncol + self.x
        reward = -1
        done = False
        if self.y == self.nrow - 1 and self.x > 0:  # 下一个位置在悬崖或者目标
            done = True
            if self.x != self.ncol - 1:
                reward = -100
        return next_state, reward, done

    def reset(self):  # 回归初始状态,坐标轴原点在左上角
        self.x = 0
        self.y = self.nrow - 1
        return self.y * self.ncol + self.x

class Sarsa:
    """ Sarsa算法 """
    def __init__(self, ncol, nrow, epsilon, alpha, gamma, n_action=4):
        self.Q_table = np.zeros([nrow * ncol, n_action])  # 初始化Q(s,a)表格
        self.n_action = n_action  # 动作个数
        self.alpha = alpha  # 学习率
        self.gamma = gamma  # 折扣因子
        self.epsilon = epsilon  # epsilon-贪婪策略中的参数

    def take_action(self, state):  # 选取下一步的操作,具体实现为epsilon-贪婪
        if np.random.random() < self.epsilon:
            action = np.random.randint(self.n_action)
        else:
            action = np.argmax(self.Q_table[state])
        return action

    def best_action(self, state):  # 用于打印策略
        Q_max = np.max(self.Q_table[state])
        a = [0 for _ in range(self.n_action)]
        for i in range(self.n_action):  # 若两个动作的价值一样,都会记录下来
            if self.Q_table[state, i] == Q_max:
                a[i] = 1
        return a

    def update(self, s0, a0, r, s1, a1):
        td_error = r + self.gamma * self.Q_table[s1, a1] - self.Q_table[s0, a0]
        self.Q_table[s0, a0] += self.alpha * td_error

ncol = 12
nrow = 4
env = CliffWalkingEnv(ncol, nrow)
np.random.seed(0)
epsilon = 0.1
alpha = 0.1
gamma = 0.9
agent = Sarsa(ncol, nrow, epsilon, alpha, gamma)
num_episodes = 500  # 智能体在环境中运行的序列的数量

return_list = []  # 记录每一条序列的回报
for i in range(10):  # 显示10个进度条
    # tqdm的进度条功能
    with tqdm(total=int(num_episodes / 10), desc='Iteration %d' % i) as pbar:
        for i_episode in range(int(num_episodes / 10)):  # 每个进度条的序列数
            episode_return = 0
            state = env.reset()
            action = agent.take_action(state)
            done = False
            while not done:
                next_state, reward, done = env.step(action)
                next_action = agent.take_action(next_state)
                episode_return += reward  # 这里回报的计算不进行折扣因子衰减
                agent.update(state, action, reward, next_state, next_action)
                state = next_state
                action = next_action
            return_list.append(episode_return)
            if (i_episode + 1) % 10 == 0:  # 每10条序列打印一下这10条序列的平均回报
                pbar.set_postfix({
                    'episode':
                    '%d' % (num_episodes / 10 * i + i_episode + 1),
                    'return':
                    '%.3f' % np.mean(return_list[-10:])
                })
            pbar.update(1)

episodes_list = list(range(len(return_list)))
plt.plot(episodes_list, return_list)
plt.xlabel('Episodes')
plt.ylabel('Returns')
plt.title('Sarsa on {}'.format('Cliff Walking'))
plt.show()
\end{lstlisting}
\section{Q-learning算法$pytorch$代码}
\begin{lstlisting}
class QLearning:
    """ Q-learning算法 """
    def __init__(self, ncol, nrow, epsilon, alpha, gamma, n_action=4):
        self.Q_table = np.zeros([nrow * ncol, n_action])  # 初始化Q(s,a)表格
        self.n_action = n_action  # 动作个数
        self.alpha = alpha  # 学习率
        self.gamma = gamma  # 折扣因子
        self.epsilon = epsilon  # epsilon-贪婪策略中的参数

    def take_action(self, state):  #选取下一步的操作
        if np.random.random() < self.epsilon:
            action = np.random.randint(self.n_action)
        else:
            action = np.argmax(self.Q_table[state])
        return action

    def best_action(self, state):  # 用于打印策略
        Q_max = np.max(self.Q_table[state])
        a = [0 for _ in range(self.n_action)]
        for i in range(self.n_action):
            if self.Q_table[state, i] == Q_max:
                a[i] = 1
        return a

    def update(self, s0, a0, r, s1):
        td_error = r + self.gamma * self.Q_table[s1].max(
        ) - self.Q_table[s0, a0]
        self.Q_table[s0, a0] += self.alpha * td_error


np.random.seed(0)
epsilon = 0.1
alpha = 0.1
gamma = 0.9
agent = QLearning(ncol, nrow, epsilon, alpha, gamma)
num_episodes = 500  # 智能体在环境中运行的序列的数量

return_list = []  # 记录每一条序列的回报
for i in range(10):  # 显示10个进度条
    # tqdm的进度条功能
    with tqdm(total=int(num_episodes / 10), desc='Iteration %d' % i) as pbar:
        for i_episode in range(int(num_episodes / 10)):  # 每个进度条的序列数
            episode_return = 0
            state = env.reset()
            done = False
            while not done:
                action = agent.take_action(state)
                next_state, reward, done = env.step(action)
                episode_return += reward  # 这里回报的计算不进行折扣因子衰减
                agent.update(state, action, reward, next_state)
                state = next_state
            return_list.append(episode_return)
            if (i_episode + 1) % 10 == 0:  # 每10条序列打印一下这10条序列的平均回报
                pbar.set_postfix({
                    'episode':
                    '%d' % (num_episodes / 10 * i + i_episode + 1),
                    'return':
                    '%.3f' % np.mean(return_list[-10:])
                })
            pbar.update(1)

episodes_list = list(range(len(return_list)))
plt.plot(episodes_list, return_list)
plt.xlabel('Episodes')
plt.ylabel('Returns')
plt.title('Q-learning on {}'.format('Cliff Walking'))
plt.show()

action_meaning = ['^', 'v', '<', '>']
print('Q-learning算法最终收敛得到的策略为:')
print_agent(agent, env, action_meaning, list(range(37, 47)), [47])
\end{lstlisting}

\section{DQN算法$pytorch$代码}
\begin{lstlisting}
import random
import gym
import numpy as np
import collections
from tqdm import tqdm
import torch
import torch.nn.functional as F
import matplotlib.pyplot as plt
import rl_utils

class ReplayBuffer:
    ''' 经验回放池 '''
    def __init__(self, capacity):
        self.buffer = collections.deque(maxlen=capacity)  # 队列,先进先出

    def add(self, state, action, reward, next_state, done):  # 将数据加入buffer
        self.buffer.append((state, action, reward, next_state, done))

    def sample(self, batch_size):  # 从buffer中采样数据,数量为batch_size
        transitions = random.sample(self.buffer, batch_size)
        state, action, reward, next_state, done = zip(*transitions)
        return np.array(state), action, reward, np.array(next_state), done

    def size(self):  # 目前buffer中数据的数量
        return len(self.buffer)

class Qnet(torch.nn.Module):
    ''' 只有一层隐藏层的Q网络 '''
    def __init__(self, state_dim, hidden_dim, action_dim):
        super(Qnet, self).__init__()
        self.fc1 = torch.nn.Linear(state_dim, hidden_dim)
        self.fc2 = torch.nn.Linear(hidden_dim, action_dim)

    def forward(self, x):
        x = F.relu(self.fc1(x))  # 隐藏层使用ReLU激活函数
        return self.fc2(x)

class DQN:
    ''' DQN算法 '''
    def __init__(self, state_dim, hidden_dim, action_dim, learning_rate, gamma,
                 epsilon, target_update, device):
        self.action_dim = action_dim
        self.q_net = Qnet(state_dim, hidden_dim,
                          self.action_dim).to(device)  # Q网络
        # 目标网络
        self.target_q_net = Qnet(state_dim, hidden_dim,
                                 self.action_dim).to(device)
        # 使用Adam优化器
        self.optimizer = torch.optim.Adam(self.q_net.parameters(),
                                          lr=learning_rate)
        self.gamma = gamma  # 折扣因子
        self.epsilon = epsilon  # epsilon-贪婪策略
        self.target_update = target_update  # 目标网络更新频率
        self.count = 0  # 计数器,记录更新次数
        self.device = device

    def take_action(self, state):  # epsilon-贪婪策略采取动作
        if np.random.random() < self.epsilon:
            action = np.random.randint(self.action_dim)
        else:
            state = torch.tensor([state], dtype=torch.float).to(self.device)
            action = self.q_net(state).argmax().item()
        return action

    def update(self, transition_dict):
        states = torch.tensor(transition_dict['states'],
                              dtype=torch.float).to(self.device)
        actions = torch.tensor(transition_dict['actions']).view(-1, 1).to(
            self.device)
        rewards = torch.tensor(transition_dict['rewards'],
                               dtype=torch.float).view(-1, 1).to(self.device)
        next_states = torch.tensor(transition_dict['next_states'],
                                   dtype=torch.float).to(self.device)
        dones = torch.tensor(transition_dict['dones'],
                             dtype=torch.float).view(-1, 1).to(self.device)

        q_values = self.q_net(states).gather(1, actions)  # Q值
        # 下个状态的最大Q值
        max_next_q_values = self.target_q_net(next_states).max(1)[0].view(
            -1, 1)
        q_targets = rewards + self.gamma * max_next_q_values * (1 - dones
                                                                )  # TD误差目标
        dqn_loss = torch.mean(F.mse_loss(q_values, q_targets))  # 均方误差损失函数
        self.optimizer.zero_grad()  # PyTorch中默认梯度会累积,这里需要显式将梯度置为0
        dqn_loss.backward()  # 反向传播更新参数
        self.optimizer.step()

        if self.count % self.target_update == 0:
            self.target_q_net.load_state_dict(
                self.q_net.state_dict())  # 更新目标网络
        self.count += 1

lr = 2e-3
num_episodes = 500
hidden_dim = 128
gamma = 0.98
epsilon = 0.01
target_update = 10
buffer_size = 10000
minimal_size = 500
batch_size = 64
device = torch.device("cuda") if torch.cuda.is_available() else torch.device(
    "cpu")

env_name = 'CartPole-v0'
env = gym.make(env_name)
random.seed(0)
np.random.seed(0)
env.seed(0)
torch.manual_seed(0)
replay_buffer = ReplayBuffer(buffer_size)
state_dim = env.observation_space.shape[0]
action_dim = env.action_space.n
agent = DQN(state_dim, hidden_dim, action_dim, lr, gamma, epsilon,
            target_update, device)

return_list = []
for i in range(10):
    with tqdm(total=int(num_episodes / 10), desc='Iteration %d' % i) as pbar:
        for i_episode in range(int(num_episodes / 10)):
            episode_return = 0
            state = env.reset()
            done = False
            while not done:
                action = agent.take_action(state)
                next_state, reward, done, _ = env.step(action)
                replay_buffer.add(state, action, reward, next_state, done)
                state = next_state
                episode_return += reward
                # 当buffer数据的数量超过一定值后,才进行Q网络训练
                if replay_buffer.size() > minimal_size:
                    b_s, b_a, b_r, b_ns, b_d = replay_buffer.sample(batch_size)
                    transition_dict = {
                        'states': b_s,
                        'actions': b_a,
                        'next_states': b_ns,
                        'rewards': b_r,
                        'dones': b_d
                    }
                    agent.update(transition_dict)
            return_list.append(episode_return)
            if (i_episode + 1) % 10 == 0:
                pbar.set_postfix({
                    'episode':
                    '%d' % (num_episodes / 10 * i + i_episode + 1),
                    'return':
                    '%.3f' % np.mean(return_list[-10:])
                })
            pbar.update(1)
\end{lstlisting}

\section{Actor-Critic算法$pytorch$代码}
\begin{lstlisting}
import gym
import torch
import torch.nn.functional as F
import numpy as np
import matplotlib.pyplot as plt
import rl_utils

class PolicyNet(torch.nn.Module):
    def __init__(self, state_dim, hidden_dim, action_dim):
        super(PolicyNet, self).__init__()
        self.fc1 = torch.nn.Linear(state_dim, hidden_dim)
        self.fc2 = torch.nn.Linear(hidden_dim, action_dim)

    def forward(self, x):
        x = F.relu(self.fc1(x))
        return F.softmax(self.fc2(x), dim=1)

class ValueNet(torch.nn.Module):
    def __init__(self, state_dim, hidden_dim):
        super(ValueNet, self).__init__()
        self.fc1 = torch.nn.Linear(state_dim, hidden_dim)
        self.fc2 = torch.nn.Linear(hidden_dim, 1)

    def forward(self, x):
        x = F.relu(self.fc1(x))
        return self.fc2(x)

class ActorCritic:
    def __init__(self, state_dim, hidden_dim, action_dim, actor_lr, critic_lr,
                 gamma, device):
        # 策略网络
        self.actor = PolicyNet(state_dim, hidden_dim, action_dim).to(device)
        self.critic = ValueNet(state_dim, hidden_dim).to(device)  # 价值网络
        # 策略网络优化器
        self.actor_optimizer = torch.optim.Adam(self.actor.parameters(),
                                                lr=actor_lr)
        self.critic_optimizer = torch.optim.Adam(self.critic.parameters(),
                                                 lr=critic_lr)  # 价值网络优化器
        self.gamma = gamma
        self.device = device

    def take_action(self, state):
        state = torch.tensor([state], dtype=torch.float).to(self.device)
        probs = self.actor(state)
        action_dist = torch.distributions.Categorical(probs)
        action = action_dist.sample()
        return action.item()

    def update(self, transition_dict):
        states = torch.tensor(transition_dict['states'],
                              dtype=torch.float).to(self.device)
        actions = torch.tensor(transition_dict['actions']).view(-1, 1).to(
            self.device)
        rewards = torch.tensor(transition_dict['rewards'],
                               dtype=torch.float).view(-1, 1).to(self.device)
        next_states = torch.tensor(transition_dict['next_states'],
                                   dtype=torch.float).to(self.device)
        dones = torch.tensor(transition_dict['dones'],
                             dtype=torch.float).view(-1, 1).to(self.device)

        # 时序差分目标
        td_target = rewards + self.gamma * self.critic(next_states) * (1 -
                                                                       dones)
        td_delta = td_target - self.critic(states)  # 时序差分误差
        log_probs = torch.log(self.actor(states).gather(1, actions))
        actor_loss = torch.mean(-log_probs * td_delta.detach())
        # 均方误差损失函数
        critic_loss = torch.mean(
            F.mse_loss(self.critic(states), td_target.detach()))
        self.actor_optimizer.zero_grad()
        self.critic_optimizer.zero_grad()
        actor_loss.backward()  # 计算策略网络的梯度
        critic_loss.backward()  # 计算价值网络的梯度
        self.actor_optimizer.step()  # 更新策略网络的参数
        self.critic_optimizer.step()  # 更新价值网络的参数

actor_lr = 1e-3
critic_lr = 1e-2
num_episodes = 1000
hidden_dim = 128
gamma = 0.98
device = torch.device("cuda") if torch.cuda.is_available() else torch.device(
    "cpu")

env_name = 'CartPole-v0'
env = gym.make(env_name)
env.seed(0)
torch.manual_seed(0)
state_dim = env.observation_space.shape[0]
action_dim = env.action_space.n
agent = ActorCritic(state_dim, hidden_dim, action_dim, actor_lr, critic_lr,
                    gamma, device)

return_list = rl_utils.train_on_policy_agent(env, agent, num_episodes)

episodes_list = list(range(len(return_list)))
plt.plot(episodes_list, return_list)
plt.xlabel('Episodes')
plt.ylabel('Returns')
plt.title('Actor-Critic on {}'.format(env_name))
plt.show()

mv_return = rl_utils.moving_average(return_list, 9)
plt.plot(episodes_list, mv_return)
plt.xlabel('Episodes')
plt.ylabel('Returns')
plt.title('Actor-Critic on {}'.format(env_name))
plt.show()
\end{lstlisting}

\section{TRPO算法$pytorch$代码}
\begin{lstlisting}
import torch
import numpy as np
import gym
import matplotlib.pyplot as plt
import torch.nn.functional as F
import rl_utils
import copy

def compute_advantage(gamma, lmbda, td_delta):
    td_delta = td_delta.detach().numpy()
    advantage_list = []
    advantage = 0.0
    for delta in td_delta[::-1]:
        advantage = gamma * lmbda * advantage + delta
        advantage_list.append(advantage)
    advantage_list.reverse()
    return torch.tensor(advantage_list, dtype=torch.float)

class PolicyNet(torch.nn.Module):
    def __init__(self, state_dim, hidden_dim, action_dim):
        super(PolicyNet, self).__init__()
        self.fc1 = torch.nn.Linear(state_dim, hidden_dim)
        self.fc2 = torch.nn.Linear(hidden_dim, action_dim)

    def forward(self, x):
        x = F.relu(self.fc1(x))
        return F.softmax(self.fc2(x), dim=1)


class ValueNet(torch.nn.Module):
    def __init__(self, state_dim, hidden_dim):
        super(ValueNet, self).__init__()
        self.fc1 = torch.nn.Linear(state_dim, hidden_dim)
        self.fc2 = torch.nn.Linear(hidden_dim, 1)

    def forward(self, x):
        x = F.relu(self.fc1(x))
        return self.fc2(x)


class TRPO:
    """ TRPO算法 """
    def __init__(self, hidden_dim, state_space, action_space, lmbda,
                 kl_constraint, alpha, critic_lr, gamma, device):
        state_dim = state_space.shape[0]
        action_dim = action_space.n
        # 策略网络参数不需要优化器更新
        self.actor = PolicyNet(state_dim, hidden_dim, action_dim).to(device)
        self.critic = ValueNet(state_dim, hidden_dim).to(device)
        self.critic_optimizer = torch.optim.Adam(self.critic.parameters(),
                                                 lr=critic_lr)
        self.gamma = gamma
        self.lmbda = lmbda  # GAE参数
        self.kl_constraint = kl_constraint  # KL距离最大限制
        self.alpha = alpha  # 线性搜索参数
        self.device = device

    def take_action(self, state):
        state = torch.tensor([state], dtype=torch.float).to(self.device)
        probs = self.actor(state)
        action_dist = torch.distributions.Categorical(probs)
        action = action_dist.sample()
        return action.item()

    def hessian_matrix_vector_product(self, states, old_action_dists, vector):
        # 计算黑塞矩阵和一个向量的乘积
        new_action_dists = torch.distributions.Categorical(self.actor(states))
        kl = torch.mean(
            torch.distributions.kl.kl_divergence(old_action_dists,
                                                 new_action_dists))  # 计算平均KL距离
        kl_grad = torch.autograd.grad(kl,
                                      self.actor.parameters(),
                                      create_graph=True)
        kl_grad_vector = torch.cat([grad.view(-1) for grad in kl_grad])
        # KL距离的梯度先和向量进行点积运算
        kl_grad_vector_product = torch.dot(kl_grad_vector, vector)
        grad2 = torch.autograd.grad(kl_grad_vector_product,
                                    self.actor.parameters())
        grad2_vector = torch.cat([grad.view(-1) for grad in grad2])
        return grad2_vector

    def conjugate_gradient(self, grad, states, old_action_dists):  # 共轭梯度法求解方程
        x = torch.zeros_like(grad)
        r = grad.clone()
        p = grad.clone()
        rdotr = torch.dot(r, r)
        for i in range(10):  # 共轭梯度主循环
            Hp = self.hessian_matrix_vector_product(states, old_action_dists,
                                                    p)
            alpha = rdotr / torch.dot(p, Hp)
            x += alpha * p
            r -= alpha * Hp
            new_rdotr = torch.dot(r, r)
            if new_rdotr < 1e-10:
                break
            beta = new_rdotr / rdotr
            p = r + beta * p
            rdotr = new_rdotr
        return x

    def compute_surrogate_obj(self, states, actions, advantage, old_log_probs,
                              actor):  # 计算策略目标
        log_probs = torch.log(actor(states).gather(1, actions))
        ratio = torch.exp(log_probs - old_log_probs)
        return torch.mean(ratio * advantage)

    def line_search(self, states, actions, advantage, old_log_probs,
                    old_action_dists, max_vec):  # 线性搜索
        old_para = torch.nn.utils.convert_parameters.parameters_to_vector(
            self.actor.parameters())
        old_obj = self.compute_surrogate_obj(states, actions, advantage,
                                             old_log_probs, self.actor)
        for i in range(15):  # 线性搜索主循环
            coef = self.alpha**i
            new_para = old_para + coef * max_vec
            new_actor = copy.deepcopy(self.actor)
            torch.nn.utils.convert_parameters.vector_to_parameters(
                new_para, new_actor.parameters())
            new_action_dists = torch.distributions.Categorical(
                new_actor(states))
            kl_div = torch.mean(
                torch.distributions.kl.kl_divergence(old_action_dists,
                                                     new_action_dists))
            new_obj = self.compute_surrogate_obj(states, actions, advantage,
                                                 old_log_probs, new_actor)
            if new_obj > old_obj and kl_div < self.kl_constraint:
                return new_para
        return old_para

    def policy_learn(self, states, actions, old_action_dists, old_log_probs,
                     advantage):  # 更新策略函数
        surrogate_obj = self.compute_surrogate_obj(states, actions, advantage,
                                                   old_log_probs, self.actor)
        grads = torch.autograd.grad(surrogate_obj, self.actor.parameters())
        obj_grad = torch.cat([grad.view(-1) for grad in grads]).detach()
        # 用共轭梯度法计算x = H^(-1)g
        descent_direction = self.conjugate_gradient(obj_grad, states,
                                                    old_action_dists)

        Hd = self.hessian_matrix_vector_product(states, old_action_dists,
                                                descent_direction)
        max_coef = torch.sqrt(2 * self.kl_constraint /
                              (torch.dot(descent_direction, Hd) + 1e-8))
        new_para = self.line_search(states, actions, advantage, old_log_probs,
                                    old_action_dists,
                                    descent_direction * max_coef)  # 线性搜索
        torch.nn.utils.convert_parameters.vector_to_parameters(
            new_para, self.actor.parameters())  # 用线性搜索后的参数更新策略

    def update(self, transition_dict):
        states = torch.tensor(transition_dict['states'],
                              dtype=torch.float).to(self.device)
        actions = torch.tensor(transition_dict['actions']).view(-1, 1).to(
            self.device)
        rewards = torch.tensor(transition_dict['rewards'],
                               dtype=torch.float).view(-1, 1).to(self.device)
        next_states = torch.tensor(transition_dict['next_states'],
                                   dtype=torch.float).to(self.device)
        dones = torch.tensor(transition_dict['dones'],
                             dtype=torch.float).view(-1, 1).to(self.device)
        td_target = rewards + self.gamma * self.critic(next_states) * (1 -
                                                                       dones)
        td_delta = td_target - self.critic(states)
        advantage = compute_advantage(self.gamma, self.lmbda,
                                      td_delta.cpu()).to(self.device)
        old_log_probs = torch.log(self.actor(states).gather(1,
                                                            actions)).detach()
        old_action_dists = torch.distributions.Categorical(
            self.actor(states).detach())
        critic_loss = torch.mean(
            F.mse_loss(self.critic(states), td_target.detach()))
        self.critic_optimizer.zero_grad()
        critic_loss.backward()
        self.critic_optimizer.step()  # 更新价值函数
        # 更新策略函数
        self.policy_learn(states, actions, old_action_dists, old_log_probs,
                          advantage)

num_episodes = 500
hidden_dim = 128
gamma = 0.98
lmbda = 0.95
critic_lr = 1e-2
kl_constraint = 0.0005
alpha = 0.5
device = torch.device("cuda") if torch.cuda.is_available() else torch.device(
    "cpu")

env_name = 'CartPole-v0'
env = gym.make(env_name)
env.seed(0)
torch.manual_seed(0)
agent = TRPO(hidden_dim, env.observation_space, env.action_space, lmbda,
             kl_constraint, alpha, critic_lr, gamma, device)
return_list = rl_utils.train_on_policy_agent(env, agent, num_episodes)

episodes_list = list(range(len(return_list)))
plt.plot(episodes_list, return_list)
plt.xlabel('Episodes')
plt.ylabel('Returns')
plt.title('TRPO on {}'.format(env_name))
plt.show()

mv_return = rl_utils.moving_average(return_list, 9)
plt.plot(episodes_list, mv_return)
plt.xlabel('Episodes')
plt.ylabel('Returns')
plt.title('TRPO on {}'.format(env_name))
plt.show()
\end{lstlisting}

\section{PPO算法$pytorch$代码}
\begin{lstlisting}
import gym
import torch
import torch.nn.functional as F
import numpy as np
import matplotlib.pyplot as plt
import rl_utils


class PolicyNet(torch.nn.Module):
    def __init__(self, state_dim, hidden_dim, action_dim):
        super(PolicyNet, self).__init__()
        self.fc1 = torch.nn.Linear(state_dim, hidden_dim)
        self.fc2 = torch.nn.Linear(hidden_dim, action_dim)

    def forward(self, x):
        x = F.relu(self.fc1(x))
        return F.softmax(self.fc2(x), dim=1)


class ValueNet(torch.nn.Module):
    def __init__(self, state_dim, hidden_dim):
        super(ValueNet, self).__init__()
        self.fc1 = torch.nn.Linear(state_dim, hidden_dim)
        self.fc2 = torch.nn.Linear(hidden_dim, 1)

    def forward(self, x):
        x = F.relu(self.fc1(x))
        return self.fc2(x)


class PPO:
    ''' PPO算法,采用截断方式 '''
    def __init__(self, state_dim, hidden_dim, action_dim, actor_lr, critic_lr,
                 lmbda, epochs, eps, gamma, device):
        self.actor = PolicyNet(state_dim, hidden_dim, action_dim).to(device)
        self.critic = ValueNet(state_dim, hidden_dim).to(device)
        self.actor_optimizer = torch.optim.Adam(self.actor.parameters(),
                                                lr=actor_lr)
        self.critic_optimizer = torch.optim.Adam(self.critic.parameters(),
                                                 lr=critic_lr)
        self.gamma = gamma
        self.lmbda = lmbda
        self.epochs = epochs  # 一条序列的数据用来训练轮数
        self.eps = eps  # PPO中截断范围的参数
        self.device = device

    def take_action(self, state):
        state = torch.tensor([state], dtype=torch.float).to(self.device)
        probs = self.actor(state)
        action_dist = torch.distributions.Categorical(probs)
        action = action_dist.sample()
        return action.item()

    def update(self, transition_dict):
        states = torch.tensor(transition_dict['states'],
                              dtype=torch.float).to(self.device)
        actions = torch.tensor(transition_dict['actions']).view(-1, 1).to(
            self.device)
        rewards = torch.tensor(transition_dict['rewards'],
                               dtype=torch.float).view(-1, 1).to(self.device)
        next_states = torch.tensor(transition_dict['next_states'],
                                   dtype=torch.float).to(self.device)
        dones = torch.tensor(transition_dict['dones'],
                             dtype=torch.float).view(-1, 1).to(self.device)
        td_target = rewards + self.gamma * self.critic(next_states) * (1 -
                                                                       dones)
        td_delta = td_target - self.critic(states)
        advantage = rl_utils.compute_advantage(self.gamma, self.lmbda,
                                               td_delta.cpu()).to(self.device)
        old_log_probs = torch.log(self.actor(states).gather(1,
                                                            actions)).detach()

        for _ in range(self.epochs):
            log_probs = torch.log(self.actor(states).gather(1, actions))
            ratio = torch.exp(log_probs - old_log_probs)
            surr1 = ratio * advantage
            surr2 = torch.clamp(ratio, 1 - self.eps,
                                1 + self.eps) * advantage  # 截断
            actor_loss = torch.mean(-torch.min(surr1, surr2))  # PPO损失函数
            critic_loss = torch.mean(
                F.mse_loss(self.critic(states), td_target.detach()))
            self.actor_optimizer.zero_grad()
            self.critic_optimizer.zero_grad()
            actor_loss.backward()
            critic_loss.backward()
            self.actor_optimizer.step()
            self.critic_optimizer.step()

actor_lr = 1e-3
critic_lr = 1e-2
num_episodes = 500
hidden_dim = 128
gamma = 0.98
lmbda = 0.95
epochs = 10
eps = 0.2
device = torch.device("cuda") if torch.cuda.is_available() else torch.device(
    "cpu")

env_name = 'CartPole-v0'
env = gym.make(env_name)
env.seed(0)
torch.manual_seed(0)
state_dim = env.observation_space.shape[0]
action_dim = env.action_space.n
agent = PPO(state_dim, hidden_dim, action_dim, actor_lr, critic_lr, lmbda,
            epochs, eps, gamma, device)

return_list = rl_utils.train_on_policy_agent(env, agent, num_episodes)

episodes_list = list(range(len(return_list)))
plt.plot(episodes_list, return_list)
plt.xlabel('Episodes')
plt.ylabel('Returns')
plt.title('PPO on {}'.format(env_name))
plt.show()

mv_return = rl_utils.moving_average(return_list, 9)
plt.plot(episodes_list, mv_return)
plt.xlabel('Episodes')
plt.ylabel('Returns')
plt.title('PPO on {}'.format(env_name))
plt.show()
\end{lstlisting}

\section{共轭梯度法解HMM模型$python$代码}
\begin{lstlisting}
import numpy as np
from matplotlib import pyplot as plt
import time
import pandas as pd
from sklearn.model_selection import train_test_split
from sklearn.preprocessing import StandardScaler

class NCG():
    def __init__(self):
        # 给定相关种类参数:
        self.N1 = 4 # 观测状态种数
        self.N2 = 4 # 隐状态种数
        self.Y_dim = 3
        self.W_dim = 4
        # 模型参数:
        self.sum_dim = self.N2*self.N1 + self.N2*self.N1*self.Y_dim + self.N2 + self.N2*self.W_dim
        self.sigma0 = np.random.normal(0, 1, (self.sum_dim, 1)) # 初始参数服从N(0,1)的正态分布
        # self.sigma0 = np.ones((self.sum_dim, 1))*0.21
        # 参数分割:
        self.mu = self.sigma0[:self.N2 * self.N1, 0].reshape((self.N1, self.N2))
        beta = self.sigma0[self.N2 * self.N1:self.N2 * self.N1 + self.N2 * self.N1 * self.Y_dim, 0]
        self.beta = beta.reshape((self.N1, self.N2, self.Y_dim))
        N1 = self.N2 * self.N1 + self.N2 * self.N1 * self.Y_dim
        N2 = self.N2 * self.N1 + self.N2 * self.N1 * self.Y_dim + self.N2
        self.gamma = self.sigma0[N1:N2, 0].reshape((self.N2, 1))
        self.alpha = self.sigma0[N2:, 0].reshape((self.W_dim, self.N2))
        # self.Pij = np.random.uniform(0, 1, (self.N1, self.N1))
        # 需要给定下面三个参数:
        self.W = np.array([[31.167], [34.125], [26.667], [27.271]])
        Pij = pd.read_excel('./4观测转换频率.xlsx', sheet_name="Sheet1", header=None)
        self.Pij = Pij.values
        print(self.Pij, self.Pij.shape)
        Y_all = pd.read_excel('./4观测转换频率 - 副本.xlsx', sheet_name="Sheet1", header=None)
        self.Y_all = Y_all.values
        print(self.Y_all, self.Y_all.shape)
        # self.Pij[0, 0] = 1
        self.Pij = self.Pij / self.Pij.max(axis=1) # 行和置为1
        # 算法参数:
        self.kmax = 200
        self.eps = 1e-6

    # P(Yi, Yj)概率计算函数
    def funPij(self, mu, beta, gamma, alpha, Yi, Yj, W, i, j):
        Pij = 0
        for K1 in range(self.N2):
            for K2 in range(self.N2):
                P_Sk1_Yi_1 = np.exp(mu[i, K1] - beta[i, K1, :].reshape((1, self.Y_dim))@Yi)
                P_Sk1_Yi_2 = 0
                for j1 in range(self.N1):
                    P_Sk1_Yi_2 = P_Sk1_Yi_2+np.exp(mu[j1, K1]-beta[j1, K1, :].reshape((1,self.Y_dim))@self.Y_all[j1,:])
                P_Sk1_Yi = P_Sk1_Yi_1 / P_Sk1_Yi_2

                P_Sk2_Sk1_1 = np.exp(gamma[K2, 0] - alpha[:, K1].T@W)
                P_Sk2_Sk1_2 = 0
                for j1 in range(self.N2):
                    P_Sk2_Sk1_2 = P_Sk2_Sk1_2 + np.exp(gamma[j1, 0] - alpha[:, K1].T @ W)
                P_Sk2_Sk1 = P_Sk2_Sk1_1 / P_Sk2_Sk1_2

                P_Sk2_Yi_1 = np.exp(mu[j, K2] - beta[j, K2, :].reshape((1, self.Y_dim)) @ Yj)
                P_Sk2_Yi_2 = 0
                for j1 in range(self.N1):
                    P_Sk2_Yi_2 = P_Sk2_Yi_2+np.exp(mu[j1, K2]-beta[j1, K2, :].reshape((1,self.Y_dim))@self.Y_all[j1,:])
                P_Sk2_Yi = P_Sk2_Yi_1 / P_Sk2_Yi_2

                Pij = Pij + P_Sk1_Yi*P_Sk2_Sk1*P_Sk2_Yi
        return Pij

    def P_Y_S_fun(self, mu, beta, gamma, alpha):
        P_Y_S = np.zeros((self.N1, self.N2))
        for i in range(self.N1):
            for j in range(self.N2):
                P_Yi_Sj_1 = np.exp(mu[i, j] - beta[i, j, :].reshape((1, self.Y_dim)) @ self.Y_all[i, :])
                P_Yi_Sj_2 = 0
                for j1 in range(self.N1):
                    P_Yi_Sj_2 += np.exp(mu[j1, j] - beta[j1, j, :].reshape((1, self.Y_dim)) @ self.Y_all[j1, :])
                P_Y_S[i, j] = P_Yi_Sj_1 / P_Yi_Sj_2
        return P_Y_S

    # P(Yi, Yj)梯度计算函数
    def gfunPij(self, mu, beta, gamma, alpha, Yi, Yj, W, i, j):
        g_mu = mu - mu
        g_beta = beta - beta
        g_gamma = gamma - gamma
        g_alpha = alpha - alpha

        for K1 in range(self.N2):
            for K2 in range(self.N2):
                P1 = np.exp(mu[i, K1] - beta[i, K1, :].reshape((1, self.Y_dim))@Yi)
                P2 = 0
                for j1 in range(self.N1):
                    P2 = P2 + np.exp(mu[j1, K1] - beta[j1, K1, :].reshape((1, self.Y_dim)) @ self.Y_all[j1, :])
                P_Sk1_Yi = P1 * (P2-P1) / P2 / P2

                P_Sk2_Sk1_1 = np.exp(gamma[K2, 0] - alpha[:, K1].T @ W)
                P_Sk2_Sk1_2 = 0
                for j1 in range(self.N2):
                    P_Sk2_Sk1_2 = P_Sk2_Sk1_2 + np.exp(gamma[j1, 0] - alpha[:, K1].T @ W)
                P_Sk2_Sk1 = P_Sk2_Sk1_1 / P_Sk2_Sk1_2

                P_Sk2_Yj_1 = np.exp(mu[j, K2] - beta[j, K2, :].reshape((1, self.Y_dim)) @ Yj)
                P_Sk2_Yj_2 = 0
                for j1 in range(self.N1):
                    P_Sk2_Yj_2 = P_Sk2_Yj_2+np.exp(mu[j1, K2]-beta[j1, K2, :].reshape((1,self.Y_dim))@self.Y_all[j1,:])
                P_Sk2_Yj = P_Sk2_Yj_1 / P_Sk2_Yj_2

                for j1 in range(self.N1):
                    if j1 == i:
                        g_mu[i, K1] = g_mu[i, K1] + P_Sk1_Yi * P_Sk2_Sk1 * P_Sk2_Yj
                        g_beta[i, K1, :] = g_beta[i, K1, :] + (-Yi)*P_Sk1_Yi * P_Sk2_Sk1 * P_Sk2_Yj
                    else:
                        P3 = np.exp(mu[j1, K1] - beta[j1, K1, :].reshape((1, self.Y_dim))@self.Y_all[j1, :])
                        P8 = (-P3*P1/P2/P2) * P_Sk2_Sk1 * P_Sk2_Yj
                        g_mu[j1, K1] = g_mu[j1, K1] + P8
                        g_beta[j1, K1, :] = g_beta[j1, K1, :] + (self.Y_all[j1, :])*(P3*P1/P2/P2)*P_Sk2_Sk1*P_Sk2_Yj

                P_Sk2_Yj = P_Sk2_Yj_1 * (P_Sk2_Yj_2-P_Sk2_Yj_1) / P_Sk2_Yj_2 / P_Sk2_Yj_2
                P_Sk1_Yi = P1 / P2

                for j1 in range(self.N1):
                    if j1 == j:
                        g_mu[j, K2] = g_mu[j, K2] + P_Sk1_Yi * P_Sk2_Sk1 * P_Sk2_Yj
                        g_beta[j, K2, :] = g_beta[j, K2, :] + (-Yj)*P_Sk1_Yi * P_Sk2_Sk1 * P_Sk2_Yj
                    else:
                        P3 = np.exp(mu[j1, K2] - beta[j1, K2, :].reshape((1, self.Y_dim))@self.Y_all[j1, :])
                        P8 = P_Sk1_Yi * P_Sk2_Sk1 * (-P3*P_Sk2_Yj_1/P_Sk2_Yj_2/P_Sk2_Yj_2)
                        g_mu[j1, K2] = g_mu[j1, K2] + P8
                        g_beta[j1, K2, :] = g_beta[j1, K2, :] + (-self.Y_all[j1, :])*P8

                P_Sk2_Yj = P_Sk2_Yj_1 / P_Sk2_Yj_2
                P_Sk2_Sk1 = P_Sk2_Sk1_1 * (P_Sk2_Sk1_2-P_Sk2_Sk1_1) / P_Sk2_Sk1_2 / P_Sk2_Sk1_2

                for j1 in range(self.N2):
                    if j1 == K2:
                        g_gamma[K2, 0] = g_gamma[K2, 0] + P_Sk1_Yi * P_Sk2_Sk1 * P_Sk2_Yj
                        P6 = W - W
                        for j2 in range(self.N2):
                            P6 = P6 + (-W)*np.exp(gamma[j2, 0] - alpha[:, K1].T @ W)
                        P7 = ((-W) * P_Sk2_Sk1_1) @ P_Sk2_Sk1_2
                        P6 = P6 @ P_Sk2_Sk1_1
                        g_alpha[:, K1] = g_alpha[:, K1] + ((P7-P6) / P_Sk2_Sk1_2[0] / P_Sk2_Sk1_2[0])*P_Sk1_Yi*P_Sk2_Yj
                    else:
                        P4 = np.exp(gamma[j1, 0] - alpha[:, K1].T @ W)
                        P5 = (-P4 * P_Sk2_Sk1_1 / P_Sk2_Sk1_2 / P_Sk2_Sk1_2) * P_Sk1_Yi * P_Sk2_Yj
                        g_gamma[j1, 0] = g_gamma[j1, 0] + P5
        return g_mu, g_beta, g_gamma, g_alpha

    # 参数粘贴函数
    def SigmaShape(self, mu, beta, gamma, alpha):
        sigma1 = mu.reshape(-1, 1)
        sigma2 = beta.reshape(-1, 1)
        sigma3 = gamma.reshape(-1, 1)
        sigma4 = alpha.reshape(-1, 1)
        return np.concatenate((sigma1, sigma2, sigma3, sigma4), axis=0)

    # 参数分割函数
    def SigmaReshape(self, sigma):
        mu = sigma[:self.N2 * self.N1, 0].reshape((self.N1, self.N2))
        beta = sigma[self.N2 * self.N1:self.N2 * self.N1 + self.N2 * self.N1 * self.Y_dim, 0]
        beta = beta.reshape((self.N1, self.N2, self.Y_dim))
        N1 = self.N2 * self.N1 + self.N2 * self.N1 * self.Y_dim
        N2 = self.N2 * self.N1 + self.N2 * self.N1 * self.Y_dim + self.N2
        gamma = sigma[N1:N2, 0].reshape((self.N2, 1))
        alpha = sigma[N2:, 0].reshape((self.W_dim, self.N2))
        return mu, beta, gamma, alpha

    # 损失函数(算法目标函数)
    def fun(self, sigma, Y, W, Pij):
        mu, beta, gamma, alpha = self.SigmaReshape(sigma)
        loss = 0
        for i in range(self.N1):
            for j in range(self.N1):
                pij = self.funPij(mu, beta, gamma, alpha, Y[i, :].T, Y[j, :].T, W, i, j)
                loss = loss + np.power(pij-Pij[i, j], 2)
        return loss

    # 损失函数梯度
    def gfun(self, sigma, Y, W, Pij):
        mu, beta, gamma, alpha = self.SigmaReshape(sigma)
        g_mu = self.mu - self.mu
        # print(g_mu)
        g_beta = self.beta - self.beta
        g_gamma = self.gamma - self.gamma
        g_alpha = self.alpha - self.alpha
        for i in range(self.N1):
            for j in range(self.N1):
                g_mu_n, g_beta_n, g_gamma_n, g_alpha_n = self.gfunPij(mu,beta,gamma,alpha,Y[i,:].T, Y[j, :].T, W, i, j)
                pij = self.funPij(mu, beta, gamma, alpha, Y[i, :].T, Y[j, :].T, W, i, j)
                g_mu = g_mu + 2*(pij-Pij[i, j])*g_mu_n
                g_beta = g_beta + 2*(pij-Pij[i, j])*g_beta_n
                g_gamma = g_gamma + 2*(pij-Pij[i, j])*g_gamma_n
                g_alpha = g_alpha + 2*(pij-Pij[i, j])*g_alpha_n
        sigma = self.SigmaShape(g_mu, g_beta, g_gamma, g_alpha)
        return sigma

    # nonlinear_cg--FR g_k^Tg_k/(g_{k-1}^Tg_{k-1})
    def FR(self, g1, g2):
        print("调用FR公式")
        return np.dot(g2.T, g2) / np.dot(g1.T, g1)

    # nonlinear_cg--PR
    def PR(self, g1, g2):
        print("调用PR公式")
        return max(0, np.dot(g2.T, (g2 - g1)) / (np.dot(g1.T, g1)))

    # x 当前迭代点 d 迭代方向 g当前迭代点的梯度 f 目标函数在当前点的函数值
    # Armijo步长规则
    def Armijo(self, sigma0, Y, W, Pij, d, g, f):
        alpha1 = 1
        sigma = 1e-3
        print("调用Armijo步长规则")
        for i in range(20):
            sigma_n = sigma0 + alpha1*d
            fn = self.fun(sigma_n, Y, W, Pij)
            if fn < f + sigma * alpha1 * np.dot(g.T, d):
                break
            alpha1 = alpha1 * 0.1
        return alpha1

    # 非线性共轭梯度算法框架
    def nonlinear_cg(self, sigma0, Y, W, Pij, searching, beta_para):
        print('---------BEGIN---------')
        start = time.time()
        x_old = sigma0
        g_old = self.gfun(x_old, Y, W, Pij)
        # print("1:", g_old)
        d_old = -g_old
        f_list = []
        for k in range(self.kmax):
            f_old = self.fun(x_old, Y, W, Pij)
            print("当前残差函数值:", f_old)
            if k % 100 == 0:
                print(f_old)
            f_list.append(f_old[0])
            g_old_norm = np.linalg.norm(g_old, 2)
            print('当前迭代第{}步,当前精度值{:.8f}'.format(k, g_old_norm))
            if g_old_norm < self.eps:
                break
            alpha = searching(x_old, Y, W, Pij, d_old, g_old, f_old)
            # 更新 x
            x_new = x_old + alpha * d_old
            # 计算当前迭代点的梯度
            g_new = self.gfun(x_new, Y, W, Pij)
            beta = beta_para(g_old, g_new)
            d_new = -g_new + beta * d_old

            #更新参数
            d_old = d_new
            g_old = g_new
            x_old = x_new
        end = time.time()
        time_use = end - start
        return x_new, f_list, k, time_use

    # 算法主框架:
    def main(self):
        nlcg_x1, nlcg_f1, nlcg_k1, nlcg_time1 = self.nonlinear_cg(self.sigma0, self.Y_all, self.W, self.Pij, \
                                                                  self.Armijo, self.FR)

        print("最优参数为:", self.SigmaReshape(nlcg_x1))
        mu, beta, gamma, alpha = self.SigmaReshape(nlcg_x1)
        print("观测概率为:", self.P_Y_S_fun(mu, beta, gamma, alpha))
        data_df = pd.DataFrame(self.P_Y_S_fun(mu, beta, gamma, alpha))  # 关键1,将ndarray格式转换为DataFrame
        # 更改表的索引
        data_df.columns = ['S1', 'S2', 'S3', 'S4']  # 将第一行的0,1,2,...,9变成A,B,C,...,J
        data_df.index = ['Y1', 'Y2', 'Y3', 'Y4']

        # 将文件写入excel表格中
        writer = pd.ExcelWriter('P_Y_S.xlsx')  # 关键2,创建名称为hhh的excel表格
        data_df.to_excel(writer, 'page_1',
                         float_format='%.5f')  # 关键3,float_format 控制精度,将data_df写到hhh表格的第一页中。若多个文件,可以在page_2中写入
        writer.save()
        print('“Armijo&FR”最优的目标函数值为{:.2f},使用时间: {:.2f}s'.format(nlcg_f1[-1], nlcg_time1))

        plt.rcParams['font.sans-serif'] = ['SimHei']  # 指定默认字体 SimHei为黑体
        plt.rcParams['axes.unicode_minus'] = False
        plt.plot(range(nlcg_k1 + 1), nlcg_f1, 'r', markersize=2, linewidth=2, label='强Wolfe+FR')
        plt.legend()
        plt.xlabel("迭代次数")
        plt.ylabel("目标函数值")
        plt.title("非线性共轭梯度迭代过程")
        plt.savefig("./figure/非线性共轭梯度法rand.png")
        plt.show()

if __name__ == '__main__':
    NCG1 = NCG()
    NCG1.main()
\end{lstlisting}

\section{模拟退火$(SA)$算法$python$代码}
%\begin{minted}[linenos]{c}
\begin{lstlisting}
import random
import matplotlib.pyplot as plt
import numpy as np
import math
import pandas as pd

num_city=30#城市总数
initial_t=120#初始温度
lowest_t=0.001#最低温度
M=150#当连续多次都不接受新的状态,开始改变温度
iteration=500#设置迭代次数
color = ['blue','red','green','orange','cyan','yellow']

location=np.loadtxt('city_location.txt')

#==========================================
#对称矩阵,两个城市之间的距离
def distance_p2p_mat():
    dis_mat=[]
    for i in range(30):
        dis_mat_each=[]
        for j in range(30):
            dis=math.sqrt(pow(location[i][0]-location[j][0],2)+pow(location[i][1]-location[j][1],2))
            dis_mat_each.append(dis)
        dis_mat.append(dis_mat_each)
   # print(dis_mat)
    return dis_mat

#计算所有路径对应的距离
def cal_newpath(dis_mat,path):
    dis=0
    for j in range(num_city-1):
        dis=dis_mat[path[j]][path[j+1]]+dis
    dis=dis_mat[path[29]][path[0]]+dis#回家
    return dis
for i_record in range(6):
    #==========================================
    #点对点距离矩阵
    dis_mat=distance_p2p_mat()
    #初始路径
    path=list(range(30))
    #初始距离
    dis=cal_newpath(dis_mat,path)
    #初始温度
    t_current=initial_t
    dis_min_record = []
    path_min_record = []

    while (t_current>lowest_t):#外循环,改变温度
        count_m=0#M的计数
        count_iter=0#迭代次数计数
        while (count_m<M and count_iter<iteration):#内循环,连续多次不接受新的状态或者是迭代多次,跳出内循环
            i=0
            j=0
            while(i==j):#防止随机了同一城市
                i=random.randint(1,29)
                j=random.randint(1,29)
            path_new=path.copy()
            path_new[i],path_new[j]=path_new[j],path_new[i]#任意交换两个城市的位置,产生新解
            #计算新解的距离
            dis_new=cal_newpath(dis_mat,path_new)
            #求差
            dis_delta=dis_new-dis
            #取0-1浮点随机数
            rand=random.random()
            #计算指数函数的值
            exp_d=math.exp(-dis_delta/t_current)
            #选择
            if dis_delta<0:
                path=path_new
                dis=dis_new
            elif exp_d>rand:
                path=path_new
                dis=dis_new
            else:
                count_m=count_m+1
            count_iter=count_iter+1
            dis_min_record.append(dis)
            path_min_record.append(path)
            # print('0最短距离:', dis_min)
            # print('0最短路径:', path_min)
        t_current=0.99*t_current#改变温度
    #外循环结束
    dis_min_test=pd.DataFrame(columns=['Y'],data=dis_min_record)
    dis_min_test.to_csv("./SAdis_record0"+str(i_record)+".csv")
    x_record = np.array(range(0, len(dis_min_test['Y']), 1))
    dis_min_test_scatter = dis_min_test['Y']
    plt.scatter(x_record, dis_min_test_scatter, s=0.05, c=color[i_record], alpha=1)
    dis_min=dis
    path_min=path
    print('最短距离:',dis_min)
    print('最短路径:',path_min)
plt.xlabel("iteration(SA)")
plt.ylabel("length(Km)")
plt.legend(['1','2','3','4','5','6'])
plt.savefig(r'./SAlength_six_maxalpha.png', dpi=300)
plt.show()
\end{lstlisting}
%\end{minted}
\section{遗传$(GA)$算法$python$代码}
\begin{lstlisting}
import random
import numpy as np
import math
import matplotlib.pyplot as plt
import pandas as pd
num_city=30#城市总数0-29
num_total=100#随机生成的初始解的总数
copy_num=70#保留的解的个数
cross_num=20#交叉解的个数
var_num=10#变异解的个数
color = ['blue','red','green','orange','cyan','yellow']
location=np.loadtxt('city_location.txt')
#print(location)

#随机生成初始解[[],[],[]...]
def generate_initial():
    initial=[]
    city=list(range(num_city))
    for i in range(num_total):
        random.shuffle(city)
        p=city.copy()
        while (p in initial):
            #print('2333')#随机了一个重复的解
            random.shuffle(city)
            p=city.copy()
        initial.append(p)
    return initial

#对称矩阵,两个城市之间的距离
def distance_p2p_mat():
    dis_mat=[]
    for i in range(30):
        dis_mat_each=[]
        for j in range(30):
            dis=math.sqrt(pow(location[i][0]-location[j][0],2)+pow(location[i][1]-location[j][1],2))
            dis_mat_each.append(dis)
        dis_mat.append(dis_mat_each)
   # print(dis_mat)
    return dis_mat

#目标函数计算,适应度计算,中间计算。适应度为1/总距离*10000
def dis_adp_total(dis_mat,initial):
    dis_adp=[]
#    dis_test=[]   
    for i in range(num_total):
        dis=0
        for j in range(num_city-1):
            dis=dis_mat[initial[i][j]][initial[i][j+1]]+dis
        dis=dis_mat[initial[i][29]][initial[i][0]]+dis#回家
#        dis_test.append(dis)        
        dis_adp_each= 10000.0/dis
        dis_adp.append(dis_adp_each)
#    print(dis_test)
    return dis_adp

def choose_fromlast(dis_adp,answer_source):
    mid_adp=[]
    mid_adp_each=0
    for i in range(num_total):
        mid_adp_each=dis_adp[i]+mid_adp_each
        mid_adp.append(mid_adp_each)
   # print(mid_adp)
    #产生0-mid_adp[num_total-1]之间的随机数
    #选择n-1<随机数<n的那个n的解,保留
    copy_ans=[]
    for p in range(copy_num):
        rand=random.uniform(0,mid_adp[num_total-1])#产生随机数
       # print(rand)
       # print(p)
        for j in range(num_total):
            if (rand<mid_adp[j]):#查找位置
                copy_ans.append(answer_source[j])
                break
            else:
                continue
    return copy_ans

#随机选择保留下来的70中的25个进行交叉
def cross_pronew(copy_ans):
    for i in range(cross_num):
        which=random.randint(0,copy_num-1)#选择对那个解交叉
        cross_list=copy_ans[which].copy()
        while (cross_list in copy_ans):
            p=random.randint(0,num_city-1)
            q=random.randint(0,num_city-1)
            cross_list[p],cross_list[q]=cross_list[q],cross_list[p]#第一次交换位置
            m=random.randint(0,num_city-1)
            n=random.randint(0,num_city-1)
            cross_list[m],cross_list[n]=cross_list[n],cross_list[m]#第二次交换位置            
        copy_ans.append(cross_list)
    cross_ans=copy_ans.copy()
    return cross_ans

#随机选择那95中的5个进行变异
def var_pronew(cross_ans):
    for i in range(var_num):
        which=random.randint(0,copy_num+cross_num-1)#选择对那个解交叉
        var_list=cross_ans[which].copy()
        while (var_list in cross_ans):
            p=random.randint(0,num_city-1)
            q=random.randint(0,num_city-1)
            var_list[p],var_list[q]=var_list[q],var_list[p]#交换位置
        cross_ans.append(var_list)
    var_ans=cross_ans.copy()
    return var_ans

for i_record in range(6):
    dis_min_record = []
    #path_min_record = []
    answer_source=generate_initial()
    dis_mat=distance_p2p_mat()
    #print(dis_mat)
    dis_adp=dis_adp_total(dis_mat,answer_source)
    adp_max_new=max(dis_adp)
    if (max(dis_adp)>10000/700):
        print('找到的最近距离是:',max(dis_adp))

    else:
        print('哎呀没找到,我再找找~')
        answer_new=answer_source
        dis_adp_new=dis_adp
        i_0 = 0
        while(adp_max_new<=10000/700):
            copy_answer=choose_fromlast(dis_adp_new,answer_new)
            cross_answer=cross_pronew(copy_answer)
            var_answer=var_pronew(cross_answer)
            answer_new=var_answer.copy()
            dis_adp_new=dis_adp_total(dis_mat,answer_new)
            adp_max_new=max(dis_adp_new)
            dis_min=10000/adp_max_new
            #print('这次是:',dis_min)
            dis_min_record.append(dis_min)
            #if i_0>=200000:break
            #else: i_0+=1

        dis_min=10000/adp_max_new
        print('终于找到你啦:',dis_min)
    dis_min_test = pd.DataFrame(columns=['Y'], data=dis_min_record)
    dis_min_test.to_csv("./GAdis_record0" + str(i_record) + ".csv")
    x_record = np.array(range(0, len(dis_min_test['Y']), 1))
    dis_min_test_scatter = dis_min_test['Y']
    plt.scatter(x_record, dis_min_test_scatter, s=0.05, c=color[i_record], alpha=1)
    print('最短距离:', dis_min)
    #print('最短路径:', path_min)
plt.xlabel("iteration(GA)")
plt.ylabel("length(Km)")
plt.legend(['1', '2', '3', '4', '5', '6'])
plt.savefig(r'./GAlength_six_maxalpha.png', dpi=300)
plt.show()


\end{lstlisting}
\section{蚁群$(ACO)$算法$python$代码}
\begin{lstlisting}
import random
import numpy as np
import math
import matplotlib.pyplot as plt
import pandas as pd

location=np.loadtxt('city_location.txt')
num_ant=200 #蚂蚁个数
num_city=30 #城市个数
alpha=1 #信息素影响因子
beta=1  #期望影响因子
info=0.1 #信息素的挥发率
Q=1 #常数

count_iter = 0
iter_max = 70
color = ['blue','red','green','orange','cyan','yellow']
#dis_new=1000
#==========================================
#对称矩阵,两个城市之间的距离
def distance_p2p_mat():
    dis_mat=[]
    for i in range(num_city):
        dis_mat_each=[]
        for j in range(num_city):
            dis=math.sqrt(pow(location[i][0]-location[j][0],2)+pow(location[i][1]-location[j][1],2))
            dis_mat_each.append(dis)
        dis_mat.append(dis_mat_each)
   # print(dis_mat)
    return dis_mat

#计算所有路径对应的距离
def cal_newpath(dis_mat,path_new):
    dis_list=[]
    for each in path_new:
        dis=0
        for j in range(num_city-1):
            dis=dis_mat[each[j]][each[j+1]]+dis
        dis=dis_mat[each[num_city-1]][each[0]]+dis#回家
        dis_list.append(dis)
    return dis_list

#==========================================
for i_record in range(6):
    #点对点距离矩阵
    dis_list=distance_p2p_mat()
    dis_mat=np.array(dis_list)#转为矩阵
    #期望矩阵
    e_mat_init=1.0/(dis_mat+np.diag([10000]*num_city))#加对角阵是因为除数不能是0
    diag=np.diag([1.0/10000]*num_city)
    e_mat=e_mat_init-diag#还是把对角元素变成0
    #初始化每条边的信息素浓度,全1矩阵
    pheromone_mat=np.ones((num_city,num_city))
    #初始化每只蚂蚁路径,都从0城市出发
    path_mat=np.zeros((num_ant,num_city)).astype(int)
    dis_min_record = []
    path_min_record = []

    #while dis_new>400:
    while count_iter < iter_max:
        for ant in range(num_ant):
            visit=0#都从0城市出发
            unvisit_list=list(range(1,30))#未访问的城市
            for j in range(1,num_city):
                #轮盘法选择下一个城市
                trans_list=[]
                tran_sum=0
                trans=0
                for k in range(len(unvisit_list)):
                    trans +=np.power(pheromone_mat[visit][unvisit_list[k]],alpha)*np.power(e_mat[visit][unvisit_list[k]],beta)
                    trans_list.append(trans)
                    tran_sum =trans

                rand=random.uniform(0,tran_sum)#产生随机数

                for t in range(len(trans_list)):
                    if(rand <= trans_list[t]):
                        visit_next=unvisit_list[t]


                        break
                    else:
                        continue
                path_mat[ant,j]=visit_next#填路径矩阵

                unvisit_list.remove(visit_next)#更新
                visit=visit_next#更新

        #所有蚂蚁的路径表填满之后,算每只蚂蚁的总距离
        dis_allant_list=cal_newpath(dis_mat,path_mat)

        #每次迭代更新最短距离和最短路径
        if count_iter == 0:
            dis_new=min(dis_allant_list)
            path_new=path_mat[dis_allant_list.index(dis_new)].copy()
        else:
            if min(dis_allant_list) < dis_new:
                dis_new=min(dis_allant_list)
                path_new=path_mat[dis_allant_list.index(dis_new)].copy()
        #print(dis_new)
        dis_min_record.append(dis_new)
        path_min_record.append(path_new)
        # 更新信息素矩阵
        pheromone_change=np.zeros((num_city,num_city))
        for i in range(num_ant):
            for j in range(num_city-1):
                pheromone_change[path_mat[i,j]][path_mat[i,j+1]] += Q/dis_mat[path_mat[i,j]][path_mat[i,j+1]]
            pheromone_change[path_mat[i,num_city-1]][path_mat[i,0]] += Q/dis_mat[path_mat[i,num_city-1]][path_mat[i,0]]
        pheromone_mat=(1-info)*pheromone_mat+pheromone_change
        count_iter += 1 #迭代计数+1,进入下一次

    #print('最短距离:',dis_new)
    #print('最短路径:',path_new)
    dis_min_test = pd.DataFrame(columns=['Y'], data=dis_min_record)
    dis_min_test.to_csv("./ACOdis_record0" + str(i_record) + ".csv")
    x_record = np.array(range(0, len(dis_min_test['Y']), 1))
    dis_min_test_scatter = dis_min_test['Y']
    plt.plot(x_record, dis_min_test_scatter, c=color[i_record], alpha=1)
    dis_min = dis_new
    path_min = path_new
    print('最短距离:', dis_min)
    print('最短路径:', path_min)
plt.xlabel("iteration(ACO)")
plt.ylabel("length(Km)")
plt.legend(['1', '2', '3', '4', '5', '6'])
plt.savefig(r'./ACOlength_six_maxalpha.png', dpi=300)
plt.show()
\end{lstlisting}

\section{禁忌搜索$(TS)$算法$python$代码}
\begin{lstlisting}
import random
import numpy as np
import math
import matplotlib.pyplot as plt
import pandas as pd
num_city=30#城市总数
table_len=20#禁忌表长度
location=np.loadtxt('city_location.txt')
taboo_table=[]
color = ['blue','red','green','orange','cyan','yellow']

#==========================================
#对称矩阵,两个城市之间的距离
def distance_p2p_mat():
    dis_mat=[]
    for i in range(30):
        dis_mat_each=[]
        for j in range(30):
            dis=math.sqrt(pow(location[i][0]-location[j][0],2)+pow(location[i][1]-location[j][1],2))
            dis_mat_each.append(dis)
        dis_mat.append(dis_mat_each)
   # print(dis_mat)
    return dis_mat

#计算所有路径对应的距离
def cal_newpath(dis_mat,path_new):
    dis_list=[]
    for each in path_new:
        dis=0
        for j in range(num_city-1):
            dis=dis_mat[each[j]][each[j+1]]+dis
        dis=dis_mat[each[29]][each[0]]+dis#回家
        dis_list.append(dis)
    return dis_list

#寻找上一个最优路径对应的所有领域解
def find_newpath(path_best):
    path_new=[]
    for i in range(1,num_city-1):
        for j in range(i+1,num_city):
            path=path_best.copy()
            path[i],path[j]=path[j],path[i]
            path_new.append(path)
    return path_new


#==========================================
for i_record in range(6):
    #点对点距离矩阵
    dis_mat=distance_p2p_mat()

    #设置初始解
    path_initial=[]
    initial=list(range(30))
    path_initial.append(initial)
    #print(path_initial)

    #加入禁忌表
    taboo_table.append(initial)

    #求初始解的路径长度
    dis_list=cal_newpath(dis_mat,path_initial)
    dis_best=min(dis_list)#最短距离
    path_best=path_initial[dis_list.index(dis_best)]#对应的最短路径方案
    #print(path_best)

    #初始期望
    expect_dis=dis_best
    expect_best=path_best

    #记录值
    dis_min_record = []
    path_min_record = []
    for iter in range(50):#迭代
        #寻找全领域新解
        path_new=find_newpath(path_best)
        #print(path_new)

        #求出所有新解的路径长度
        dis_new=cal_newpath(dis_mat,path_new)
        #print(dis_new)

        #选择路径
        dis_best=min(dis_new)#最短距离
        #print(dis_best)
        path_best=path_new[dis_new.index(dis_best)]#对应的最短路径方案
        dis_min_record.append(dis_best)
        path_min_record.append(path_best)
        if dis_best < expect_dis:#最短的<期望
            expect_dis=dis_best
            expect_best=path_best#更新两个期望
            if path_best in taboo_table:
                taboo_table.remove(path_best)
                taboo_table.append(path_best)
            else:
                taboo_table.append(path_best)
        else:#最短的还是不能改善期望
            if path_best in taboo_table:#在禁忌表里
                dis_new.remove(dis_best)
                path_new.remove(path_best)
                dis_best=min(dis_new)#求不在禁忌表中的最短距离
                path_best=path_new[dis_new.index(dis_best)]#对应的最短路径方案
                taboo_table.append(path_best)
            else:#不在禁忌表
                taboo_table.append(path_best)
        if len(taboo_table)>=table_len:
            del taboo_table[0]
    dis_min_test = pd.DataFrame(columns=['Y'], data=dis_min_record)
    dis_min_test.to_csv("./TS_dis_record0" + str(i_record) + ".csv")
    x_record = np.array(range(0, len(dis_min_test['Y']), 1))
    dis_min_test_scatter = dis_min_test['Y']
    plt.plot(x_record, dis_min_test_scatter, c=color[i_record], alpha=1)

    print('最短距离',expect_dis)
    print('最短路径:',expect_best)
plt.xlabel("iteration(TS)")
plt.ylabel("length(Km)")
plt.legend(['1','2','3','4','5','6'])
plt.savefig(r'./TSlength_six_maxalpha.png', dpi=300)
plt.show()
\end{lstlisting}
% \end{appendixs}










