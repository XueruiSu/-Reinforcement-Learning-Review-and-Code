%!TEX root = ../csuthesis_main.tex
% 设置中文摘要
\keywordscn{最优化理论\quad 最优化算法\quad 人工智能\quad 凸优化\quad 启发式算法\quad }
%\categorycn{TP391}
\begin{abstractzh}

最优化理论与统计分析、模拟仿真并称为数学建模的三大基本方法和技术。在应用数学知识解决实际问题时,最优化理论是非常活跃的研究发现和分支。从上世纪后半叶开始迅速发展至今。已具备非常完善的知识体系。针对不同的问题,新的理论和方法不断被提出。

作者注意到最优化理论自诞生以来,在各个领域的应用非常广泛,尤其在人工智能领域,在控制、预测、搜索、估计等任务中发挥了至关重要的作用。本文基于本学期学到的知识,站在优化的角度上,梳理了人工智能领域控制、预测、搜索三大任务情景下最优化算法的设计与使用。可以说,最优化的问题形式在这三类任务情景中均发挥了或是作为算法重要部分或是引导出更简单近似方法的重要作用。

首先,作者对最优化的知识进行了系统地梳理。解释了相关的基本概念。对一阶、二阶、带约束、无约束最优性条件进行了阐述。列举了学过的几类算法的算法框架和理论性质。然后阐述了控制、预测、搜索三类任务情景下最优化算法的设计与使用。

第一,当前在控制领域-强化学习的研究中,基本目标是在马氏决策过程的框架内,基于\textbf{贝尔曼最优方程},通过\textbf{动态规划}、\textbf{时序差分}和\textbf{策略梯度}等优化算法求解最优值函数和最优策略。由于策略函数和值函数均为复杂的抽象函数,不可直接求解,所以强化学习的目标就是寻找合适的方式不断优化策略和价值函数使其接近最优策略和“真实”的价值函数。\textbf{所以说,强化学习的算法可以看作是在特定任务情景下的优化算法。} 本文主要介绍了如何使用贝尔曼最优方程求解最优值函数和最优策略。同时对强化学习这门学科诞生三十年来(1990-2018)产生的几大算法(Q-learning、DQN、AC、TRPO、PPO等)进行了简要介绍,指出算法中“优化”思想的体现,以此来说明优化在该领域中的重要性。

第二,(此部分为作者原创内容,若有错误之处请老师指正)本文介绍了一类对直播间行为进行预测的HMM模型。但随着参数和状态空间的增大,通过极大似然对模型参数进行估计的方法是行不通的。于是作者在该框架下将问题转化为优化问题,使用非线性共轭梯度法进行求解,在实际数据上取得了较好的成果。

第三,在各种建模比赛中,启发式算法使用频率很高。这些算法不同于传统的优化算法,由于其本身的智能性,该算法被广泛应用在搜索任务情景中。本文对几种启发式优化算法进行了理论介绍,并将其应用在\textbf{旅行商问题}(\textbf{TSP})的求解中。传统搜索算法对于该问题大多是病态的,但是本文介绍的四种启发式算法却可以高效的求解该问题。

\end{abstractzh}